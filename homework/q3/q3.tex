\documentclass[12pt]{article}

\usepackage[utf8]{inputenc}

% Disable paragraph indentation
\usepackage{graphicx}

% For hyperlinks in figure captions
\usepackage{hyperref}

% Disable paragraph indentation
\setlength{\parindent}{0pt}

% Set the length of space between paragraphs
\setlength{\parskip}{12pt}

% Disable page numbering
\pagenumbering{gobble}

% For forcing figure position
\usepackage{float}


\title{Internet of Things - Homework - Question 3}
\author{Francesco Pastore 10629332}
\date{05-14-2024}

\begin{document}

\maketitle

\textbf{Question}

You are required to design an IoT system to monitor the status of the production process in a small indoor bacterial cellulose factory.
The factory is operated in a small university lab (100sqm) and has about 20 bacterial cellulose growing basins, which must be monitored continuously to ensure the growing process is successful.
The main parameters to be monitored are luminosity (2 bytes), content of sugar (2 bytes) and pH of the growing solution (1 byte).
Growing bacterial cellulose is a slow process, with growing cycles of about 14 days.
Monitoring cycles of 1 hour are needed to allow for changing the environmental parameters for an optimal process.

3. As an add-on, you are required to install a VGA camera (640x480 pixels, 8 bits per pixel) to monitor the status of the growing process. Is the solution proposed at the previous points still valid? If not, propose an alternative solution.

\textbf{Answer}

The use of WiFi technology for communication between the sensors and the central unit allows high data rates, making it suitable even if a VGA camera is added to the system.

MQTT is also still a valid choice for the application protocol, as it still allows data to be sent from the sensors to the central server.

\end{document}