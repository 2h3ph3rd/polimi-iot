\documentclass[12pt]{article}

\usepackage[utf8]{inputenc}

% Disable paragraph indentation
\usepackage{graphicx}

% For hyperlinks in figure captions
\usepackage{hyperref}

% Disable paragraph indentation
\setlength{\parindent}{0pt}

% Set the length of space between paragraphs
\setlength{\parskip}{12pt}

% Disable page numbering
\pagenumbering{gobble}

% For forcing figure position
\usepackage{float}


\title{Internet of Things - Homework - Question 1}
\author{Francesco Pastore 10629332}
\date{05-14-2024}

\begin{document}

\maketitle

\textbf{Question}

You are required to design an IoT system to monitor the status of the production process in a small indoor bacterial cellulose factory.
The factory is operated in a small university lab (100sqm) and has about 20 bacterial cellulose growing basins, which must be monitored continuously to ensure the growing process is successful.
The main parameters to be monitored are luminosity (2 bytes), content of sugar (2 bytes) and pH of the growing solution (1 byte).
Growing bacterial cellulose is a slow process, with growing cycles of about 14 days.
Monitoring cycles of 1 hour are needed to allow for changing the environmental parameters for an optimal process.

1. Propose an overall design for the system, mainly focusing on the communication technology to be used. Motivate your choice.

\textbf{Answer}

Each sensor could be made with an ESP32 board for its low cost and ability to connect to the WiFi network.

If possible, they could be powered directly with a continuous power supply to avoid the need for batteries.
Otherwise, a battery could be used, which would need to be checked and replaced if necessary.

From the communication point of view, since the required IoT system will operate in a small indoor area, I would suggest using short-range communication technology.

In particular, to simplify the design and implementation of the system, I would suggest using WiFi technology for communication between the sensors and the central server.
This would make it easier to implement the system and directly use the WiFi capabilities of the ESP32 cards.

The application protocol to be used could be MQTT, with all the sensors as different publishers and the central unit as a subscriber.
This would allow for a simple and efficient way to send data from the sensors to the central server.

The MQTT broker could run directly on the central server, which could also be used to store data and provide a web interface to monitor the status of the whole system.

\end{document}